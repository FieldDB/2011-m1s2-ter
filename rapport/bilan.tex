\documentclass[a4paper,11pt,french]{article}

\usepackage[frenchb]{babel}
\usepackage[T1]{fontenc}
\usepackage[utf8]{inputenc}
\usepackage{verbatim}
\usepackage{graphicx}
\usepackage{hyperref}

\hypersetup{%
  colorlinks,%
  linkcolor=black,%
  filecolor=black,%
  urlcolor=black,%
  citecolor=black%
}

\title{Bilan du groupe 42\\---\\Reconception du jeu Pticlic sous Android}
\author{Yoann \textsc{Bonavero} \and Bertrand \textsc{Brun} \and John \textsc{Charron} \and Georges \textsc{Dupéron} \\\\ Encadrant: M. Mathieu \textsc{Lafourcade}}

\begin{document}

\maketitle

Notre projet \og{}Reconception du jeu Pticlic sous Android\fg{} été la création d'un ou plusieurs prototype(s) sous Android\texttrademark.
Nous avons, à leur actuelle, réalisé 2 prototypes.
\\

Nous avons donc commencé par utilisé l'API fourni par Google, nommé Android\texttrademark. Nous avons développé une application fonctionnelle utilisant cette API. Cet framework utilise le langague de programmation Java. Google a mis en place un plugin pour Eclipse nomme ADT\footnote{Android Development Tools} que nous avons donc utilisé. Une fois celle-ci fini et debogger par nos soin nous avons réaliser une alpha-test que nous avons envoyé à differentes connaissances. Une fois les retours des alpha-testeur nous avons donc commencé à réaliser une second prototype incluant les differentes remarque de ces testeurs (que nous remercions de leur attentions).

Pour l'alpha-test, nous avons aussi réalisé un site web (\url{www.pticlic.fr}) pour que les différents testeur puissent s'inscrire et télécharger l'application sur leur téléphone.
\\

Pour le deuxième prototype nous avons radicalement changé la façon dont nous allions réaliser les differentes vue necéssaire pour mener à bien notre projet. La raison de ce changement est la difficulté rencontré par le groupe pour créer et modifier les vue avec l'API Android\texttrademark. Donc, nous nous sommes dirigé vers un developpement entiere en HTML5/PHP/Javascript, car nous avions plus de facilité de developpement avec ces outils qu'avec l'API proposé par Google. Pour que notre application affiche correctement notre application web, nous utilisons WebKit, fourni par Android\texttrademark, qui est un moteur de rendu HTML, ainsi qu'un moteur Javascript. L'un des avantages de l'utilisation des technologies du web, et qu'il rend notre application multi-platforme, en effet notre application sera fonctionnel sous Android mais aussi sur ordinateur traditionnel et sous iPhone (en utilisant le navigateur web).

De plus, nous avons réalisé, sur notre site web, une interface pour que les joueurs puissent créer eux-même des parties (avec un mot centrale et autant de mots dans le nuage) en fonction de son avancement dans le jeu.
\\

\noindent Il nous reste donc à réaliser~:
\begin{itemize}
  \item L'authentification dans l'application des utilisateurs;
  \item d'attributer, à un joueur, le droit de crée un partie;
  \item à afficher, dans la page de score, la personne qui a créé la partie.
\end{itemize}

\end{document}
