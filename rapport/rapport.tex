\documentclass[a4paper,11pt,french]{article}

\usepackage[frenchb]{babel}
\usepackage[T1]{fontenc}
\usepackage[utf8]{inputenc}
\usepackage{um2/um2}

\setlength{\parindent}{0pt}
\setlength{\parskip}{2ex}

\title{Rapport de TER\\---\\Reconception du jeu Pticlic sous Android}
\author{Yoann \textsc{Bonavero} \and Bertrand \textsc{Brun} \and John \textsc{Charron} \and Georges \textsc{Dupéron}}

\begin{document}

\maketitle


\pagestyle{empty}
\thispagestyle{empty}

\tableofcontents


\pagestyle{empty}
\thispagestyle{empty}
\newpage
\setcounter{page}{1}
\pagestyle{plain}


\section{Introduction}

PtiClic\footnote{http://pticlic.org} est un jeu qui a été conçu et développé par Matthieu Lafourcade et Virginie Zampa. Le jeu a été créé afin de faire des études sur le vocabulaire et la sémantique sur des sujets de divers horizons dans un contexte ludique et motivant. Un mot central apparait, un nuage de mots entoure le mot central et le joueur clique et dépose des mots du nuage dans des catégories proposé sous forme d'énoncés. 

Par exemple, pour le mot central 'bicyclette', les mots 'pédale', 'piéton', 'pied', 'automobile', 'Sébastien Chabal', 'Lance Armstrong', 'pédalier', 'voiture', 'yeux', 'rapide', 'routier', 'maillot', 'pédaler', 'dopage', 'véhicule', 'musclé', 'nez', etc. sont proposé. Le joueur dépose ces mots dans les catégories "... est une partie de 'cycliste', "Un contraire de 'cycliste' est ...", "'cycliste' a un rapport avec ...",  "Une caractéristique de 'cycliste' est ..." ou aucune de ces catégorie. Un score est obtenu en soustrayant les mots manquants et les mots incorrects des mots corrects. 

Des linguistes et des informaticiens récupèrent les données liées aux parties jouées, ce qui leur permet de faire de la recherche dans leurs domaines respectifs.

Notre travail consiste à créer une version du PtiClic sous Android, une version modifiée du jeu adaptée pour téléphone mobile. Le sujet du TER définit clairement l'objectif de ce projet~:

\begin{quotation}
%% Correction Bertarnd remplacement de fonctionnant sur des Android semble est intéressante => fonctionnant sur Android semble intéressante
L'étude et le prototypage d'une version fonctionnant sur Android semble intéressante. En particulier on s'intéressera a deux aspects : * les contraintes imposées par l'environnement smartphone * le biais qu'imposent ces contraintes sur le jeu et les données récoltées. Il s'agira donc de modéliser une version adaptée aux smartphones et d'en implémenter un prototype fonctionnel. 
\end{quotation}

Dans un premier temps, une version de base sera conçue et réalisée. Ensuite, des fonctionnalités supplémentaires seront ajouter. La démarche adoptée par notre groupe est une approche itérative. Les quatres livraisons vont d'une version de base vers des versions plus élaborées~: un joueur pourrait, entre autres, modifier ses préférences ou choisir son niveau. L'idée est aussi de rendre le jeu plus attirant afin d'accroître le nombre de sujets participant aux études liées au résultat des données extraits des parties jouées.


\subsection{Android}

Android est un système d'exploitation pour téléphone mobile basé sur le noyau Linux développé par Android Inc., racheté par Google en 2005. Google et d'autres membres du Open Handset Alliance ont par la suite contribué à son développement et le Android Open Source Project (AOSP) est chargé de la maintenance et l'évolution d'Android. Ce système d'exploitation est utilisé notamment dans des smartphones, appelé aussi ordiphones, 'terminaux de poche' ou 'téléphones intelligents', produits et distribués par un grand nombre de fabriquants de téléphones mobiles. Le nombre de téléphones mobiles intégrant le système d'exploitation d'Android a cru sensiblement récemment.

Un grand nombre de développeurs ont créés des applications pour étendre la fonctionnalité des téléphones sous Android et il y a aujourd'hui plus de 200 000 applications disponibles. Bien qu'Android Market est le magasin en ligne opéré par Google, il existe d'autres distributeurs d'applications Android. La majorité des applications sont écrites en Java, bien qu'il soit possible de développer des applications en Python, en Ruby et d'autres par le biais du Android Scripting Environment. 

 
\section{Analyse de l'existant}

\section{Analyse des besoins}

\section{Conception}

\begin{verbatim}
NODE(EID integer primary key autoincrement, name string, #type (ref TYPE_NODE.num), weight);

RELATION(RID integer primary key autoincrement, #start (ref NODE.eid), #end (ref NODE.eid), #type (ref TYPE_RELATION.num), weight);

TYPE_NODE(NUM, name string);

TYPE_RELATION(NUM, name string, extended_name string, info string);

USER(LOGIN string primary key, mail string, hash_passwd string (md5sum du password), #score (contrainte : somme de tous les scores des PLAYED_GAME_CLOUD);

GAME(GID integer primary key autoincrement, #eid_central_word (ref NODE.eid, #relation_1 (ref RELATION.rid), #relation_2 ( (ref RELATION.rid), difficulty);

GAME_CLOUD(GID, NUM, difficulty, #eid_word(ref NODE.eid), totalWeight (contrainte : = somme des probas), probaR1 (contrainte : = somme des probas des PLAYED_GAME_CLOUD.weight avec la bonne relation et la même gid et num), probaR2 (idem), probaR0 (idem), probaTrash (idem));

PLAYED_GAME(PGID integer primary key autoincrement, #gid (ref GAME.gid), #login (ref USER.login);

PLAYED_GAME_CLOUD(#PGID (ref PLAYED_GAME.pgid), #GID (ref PLAYED_GAME.gid), NUM, type (contrainte : 0 = partie de référence, 1 = réponse d'un joueur), #relation (ref RELATION.rid), weight (contrainte : probabilité estimée de cette réponse pour les bots (robots), réputation du joueur sinon), score (score donné au joueur, 0 pour les bots);

**INT unless otherwise marked
\end{verbatim}
\begin{verbatim}
create table node(eid integer primary key autoincrement, name, type, weight);
create table relation(rid integer primary key autoincrement, start, end, type, weight);
create table type_node(name, num);
create table type_relation(name, num, extended_name, info);
create table user(login primary key, mail, hash_passwd, score);
create table game(gid integer primary key autoincrement, eid_central_word, relation_1, relation_2, difficulty);
create table game_cloud(gid, num, difficulty, eid_word, totalWeight, probaR1, probaR2, probaR0, probaTrash);
create table played_game(pgid integer primary key autoincrement, gid, login);
create table played_game_cloud(pgid, gid, type, num, relation, weight, score);

create index i_relation_start on relation(start);
create index i_relation_end on relation(end);
create index i_relation_type on relation(type);
create index i_relation_start_type on relation(start,type);
create index i_relation_end_type on relation(end,type);
\end{verbatim}

TODO: UML, diagrammes de classes, Use cases, etc...


\section{Réalisation}
\subsection{Cahier des charges}
\subsubsection{Langages de programmation}
\subsubsection{Base de données}
\subsubsection{d'autres subsubsections?}

\subsection{Outils utilisés}
\subsubsection{Environnement intégré de développement~: Eclipse}
\subsubsection{Android Developper Toolkit (ADT) Plugin}
\subsubsection{Android Software Development Kit (SDK)}
\subsubsection{Gestionnaire de version~: GitHub}
\subsubsection{JUnit}
\subsubsection{d'autres subsubsections ?}


\section{Discussion}
\subsection{Difficultés rencontrées}
\subsubsection{Itération 1, semaine 1}
\begin{itemize}
\item Outil de création de diagrammes de GANTT (planner) est assez mauvais.
\item Lenteur de l'émulateur Android : impossible de travailler sur mon PC.% gd
\item Caractères non échappés dans le dump de la base.% gd
\end{itemize}

\subsubsection{Itération 1, semaine 3}
\begin{itemize}
\item SQLite3 n'est pas capable d'utiliser un index pour la requête extérieure sur une requête du type
\begin{verbatim}
select * from (select * from table where condition) where condition
\end{verbatim}
Donc nécessité de ré-écrire certaines requêtes avec des jointures à priori beaucoup moins efficaces, mais qui le sont plus grâce aux index.
\item SQLite3 tranforme les requêtes de la forme~:
\begin{verbatim}
select * from table limit 100 order by random();
\end{verbatim}
  en une requête qui récupère tout le set de résultats, ajoute une colonne random(), prend les 100 premiers résultats et les trie. Mais cela
  l'oblige à récupérer tout le set de résultats, et calculer le random() pour chaque ligne, pour ensuite jeter tout ce qui dépasse la ligne
  100. Cela est évidemment très coûteux dans le cadre de requêtes avec beaucoup de résultats, et nous avons donc dû isoler la requête avec
  \verb!limit! de son \verb!order by! avec des «hacks» assez tordus.
\end{itemize}

\section{Conclusions}

\newpage


\section{Bibliographie}
\subsection{PtiClic}

PtiClic : a game for vocabulary assessment combining JeuxDeMots and LSA. In proc of CICLing (Conference on Intelligent text processing and Comptational Linguistics). Mexico : 1-7 mars 2009. (\url{http://www.cicling.org/2009/RCS-41/289-298.pdf})


\subsection{Linguistique}

Modelling, Detection and Exploitation of Lexical Functions for Analysis , ECTI Journal, 2007, Vol.2, No2, ISSN 1905-050X, pp 97-108. (\url{http://www.lirmm.fr/\%7Eschwab/Publications/SL_ECTI_journal.pdf})

Making people play for Lexical Acquisition. In Proc. SNLP 2007, 7th Symposium on Natural Language Processing. Pattaya, Thailande, 13-15 December 2007. (\url{http://www.lirmm.fr/~lafourcade/ML-biblio/SNLP07/MLF-snlp2007-v5.doc})


\subsection{Java}

Code Conventions for the Java Programming Language, Oracle, 1999. (\url{http://www.oracle.com/technetwork/java/codeconvtoc-136057.html, www.oracle.com/technetwork/java/codeconventions-150003.pdf})

\subsection{Android}

Android Developer, 2011. (\url{http://developer.android.com/})




\section{Notes Georges}
Les relations suivantes seront peut-être utilisées (* = oui, c'est sûr, on a/doit faire les icônes et des requêtes sql)~:

\begin{tabular}{|c|l|l|l|}
\hline
icône~? & nom & num & signification \\
\hline
$*$ & r\_syn       & 5  & synonyme (chat -> matou) \\
$*$ & r\_anto      & 7  & antonyme (bon -> mauvais) \\
$*$ & r\_has\_part & 9  & A comme partie (chat -> patte) \\
$*$ & r\_holo      & 10 & Fait partie de (patte -> chat) \\
    & r\_agent     & 13 & Peut exécuter comme action (chat -> manger) \\
    & r\_patient   & 14 & Peut subir comme action (chat -> être lavé) \\
    & r\_carac     & 17 & Caractéristique (chat -> affectueux (ou pas…)) \\
\hline
\end{tabular}
\newpage

\appendix

\section{Annexe A}


\subsection{14 janvier 2010}


Durée du projet 4 mois (4 itérations de 4 semaines)

Conventions de code : \url{http://java.sun.com/docs/codeconv/html/CodeConventions.doc6.html}

Code (noms de variables, etc.) en anglais, commentaires en français, javadoc en français.

\subsection{26 janvier 2011}
Mettre le serveur (PHP) sur free.fr, pour pouvoir tester facilement

Utilisation d'une classe \verb!Constant!

Écran d'accueil du jeu : Image (splash), puis directement les icônes des modes de jeu + configuration, au lieu d'avoir un écran avec le logo et jouer/config, suivi du choix du mode de jeu.

\section{Annexe B}

\end{document}

